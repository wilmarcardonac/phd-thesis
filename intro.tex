\addcontentsline{toc}{chapter}{Introduction}
\chapter*{Introduction}
\label{intro} 

The $20$th century gradually saw the emergence of the standard model of cosmology. A linear relation between distances and recession velocities of galaxies \commentr{Hubble's paper}, the observed abundance of chemical elements in the universe \commentr{Gamow et al.  paperr}, and the existence of the Cosmic Microwave Background radiation (CMB) \commentr{Penzias et al. and Gamow or ALpher et al. D. Fixsen et al (COBE)} evidenced a dynamical rather than static universe: The universe is expanding. Observations of type Ia supernovae in $1998$ \commentr{Riess et al. S. Perlmutter et al.} modified a ``little bit'' the picture setting one the most important problems cosmologists will be addressing in the $21$st century: The universe is not only expanding, it is speeding up and cosmologists want to find out why. 

The cosmological principle -- the assumption that the universe at sufficiently large scales is homogeneous and isotropic -- is one of the cornerstones of the concordance model of cosmology \commentr{Robertson, Walker}. If one assumes there is no charge asymmetry in the universe \commentr{Chiara Caprini}, the only relevant interaction on large scales is gravity. In the vanilla model of cosmology the gravitational interaction is described by Einstein General Relativity. Solutions for Einstein's field equations -- that couple geometry to both matter-energy and pressure -- satisfying the cosmological principle are known since early $20$th century \commentr{Friedmann, Lemaitre}. In those solutions -- the so-called FLRW metric -- the expansion of the universe is given by the scale factor $a(t)$, a function which depends on the cosmic time $t$ and scales the distance between two given points as the universe expands. The matter content in the standard model of cosmology is only partially given by particles in the Standard Model (SM) of particle physics (i.e., photons, electrons, and so on). The remaining matter, dubbed Cold Dark Matter (CDM) because it only seems to interact with the baryonic matter through gravity, is required, for instance, to fit observations of galaxy velocities in galaxy clusters \commentr{Zwicky}. Finally, in order to describe the accelerated expansion of the universe, the concordance model of cosmology reintroduces the cosmological constant $\Lambda$ (first introduced by Albert Einstein in the early $20$th century). The standard model of cosmology is thus named  $\Lambda CDM$ model.

The universe is obviously not completely homogeneous as there exist inhomogeneities such as galaxies and clusters of galaxies. Moreover, both the fact that the CMB looks incredibly uniform in regions now causally disconnected and the convergence of observations into a flat universe make the $\Lambda CDM$ model an incomplete description of the universe. These three main difficulties of the standard model of cosmology (i.e., structure formation, horizon, flatness) can be solved by adding an inflationary epoch to the history of the universe \commentr{Guth and Kaiser review}. In the simplest inflationary scenarios the potential energy of a scalar field drives an exponential expansion in the very early universe rendering the universe extremely flat very quickly (within about $10^{-35}\, \second$). The universe would have evolved from a tiny patch (about $10^{-26}\,\mathrm{m}$) where regions which today are causally isolated were then in causal contact thus solving the horizon problem of the standard model of cosmology. Perhaps most importantly, inflationary models predict that quantum fluctuations of the scalar field in the early stage of the universe would have seeded the density fluctuations we observe today in the form of galaxies, galaxy clusters, and CMB fluctuations.

Over the past three decades cosmology has witnessed the come of an age of precision. Full sky CMB experiments such as COBE \commentr{Smoot}, WMAP \commentr{Bennett}, and PLANCK \commentr{Ade et al.} have measured both CMB spectrum and CMB anisotropies in different frequencies and on a wide range of angular scales allowing thus a careful study of the predictions made by inflationary $\Lambda CDM$ model. The CMB spectrum matches incredibly well that of black body with temperature $2.7\,\mathrm{K}$ as predicted by \commentr{Gamow or one of his students}. Investigating CMB fluctuations cosmologists have been able to constraint to high accuracy the curvature of the universe: It agrees pretty well with the flatness prediction of inflationary models \commentr{Planck paper cosmological parameters}. Although a bit controversial, there is no compelling evidence for significant deviations of the cosmological principle in the Planck CMB data \commentr{Planck paper about statistical isotropy}. Furthermore, current CMB experiments support the existence of dark matter and also the accelerated expansion of the universe. Galaxy surveys such as SDSS have also played an important role in testing cosmological models. Partially mapping the distribution of mass in the universe, using observations of $\approx 10^6$ galaxies with mean red shift $z \approx 0.1$, SDSS collaboration detected a baryon acoustic peak which is an imprint of the recombination-epoch acoustic oscillations on the low-redshift clustering of matter \commentr{SDSS paper Eisenstein et al.}. This detection confirms a prediction of the standard cosmological theory. Upcoming galaxy surveys offer thus an important complementary probe and will be key for testing cosmological models in the near future, for instance to determine the neutrino mass.

The current concordance cosmological model, although both simple and providing a good fit for the current data sets, lacks in fundamental grounds. On the one hand, it assumes the existence of dark matter which thus far has not been directly observed; the only known dark matter candidate being neutrino. On the other hand, the cosmological problem poses a serious conundrum for quantum field theory which is unable to explain the extremely tiny observed value of the vacuum energy. Numerous alternative approaches to explain the accelerated expansion of the universe from first principles have been proposed over the past years. 

 but a lot of unsolved questions ...inflation. Many both inflationary and background models. 
Need efficient ways to break degeneracies at the model level... massive neutrinos only known dark matter candidates ... modified gravity and dark energy and higher dimensions...   \\

In one project about anisotropic dark energy...: What people have done thus far and what those approaches lack.  , anisotropic stress key observable, what is the problem ?, why what I do is important to solve it ?, collaborators, cmb data mainly\\

In the project lensing bias: another kind of data with different systematics and physics, collaborators, neutrino mass, dark matter, reliable forecasts, careful analysis including relevant relativistic effects, what is the problem ?, why what I do is important to solve it ? What people have done thus far and what those approaches lack.  \\

Finally, in the project about Hubble constant ... reliable and accurate constraints, break degeneracies, collaborators, model independent constraints, what is the problem ?, why what I do is important to solve it ? What people have done thus far and what those approaches lack.  \\ 

How is the thesis organised ? Give a brief description of each chapter. In particular, highlight our main results and maybe point to publications (if any).


 