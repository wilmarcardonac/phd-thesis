\chapter{Basic expressions for the Fisher analysis}
\label{appendix2-mnu}

The Fisher information matrix $F_{\alpha\beta}$ is given by
\begin{equation}
F_{\alpha\beta} =
\sum_{\ell=2}^{\ell_{\mathrm{max}}} \sum_{(ij)(pq)} \frac{\partial C_\ell^{ij}}{\partial \theta_\alpha}
\frac{\partial C_\ell^{pq}}{\partial
\theta_\beta} {{\rm Cov}_{C_{\ell \, \rm[(ij), (pq)]}}^{-1}} \, ,
\label{eq:Fisher}
\end{equation}
where $\theta_a$ denotes a given cosmological parameter, $C_\ell^{ij}$ is the number counts angular power spectrum of Eq. \eqref{Eq:binned-angular-matter-power-spectrum}, and the covariance matrix ${{\rm Cov}_{C_{\ell \, \rm[(ij), (pq)]}}^{-1}}$ is
\begin{equation}
\label{eq:err-clgt}
{\rm Cov}_{C_{\ell \, \rm[(ij), (pq)]}} = \frac{C^{\rm A, (ip)}_{\ell} C^{\rm A, (jq)}_{\ell} + C^{\rm A, (iq)}_{\ell} C^{\rm A, (jp)}_{\ell}}{(2\ell+1)f_{\rm sky}},
\end{equation}
where $A=\mathrm{obs},\,\mathrm{th}$; $i,j,p,q = 1,...,N_{\mathrm{bin}}$; and $f_{\mathrm{sky}}$ is the covered sky fraction. The derivatives in Eq. \eqref{eq:Fisher} are computed with a five-point stencil \cite{Montanari:2015rga}; we  choose the step for each parameter with an iterative procedure: the step is chosen to be of the same size as the 1-$\sigma$ widths obtained when fixing the other parameters $\sigma_{\theta_{\alpha}}=1/\sqrt{F_{\alpha\alpha}}$.
We have verified that our results are not significantly affected by the particular step values. Since we treat non-linearities in a conservative way, we sum up to $\ell_{\mathrm{max}}=400$. The second sum in Eq. \eqref{eq:Fisher} is over the couple of matrix indices $(ij)$ and $(pq)$ with $i \leq j$ and $p \leq q$ which run from $1$ to the total number of bins $N_{\mathrm{bin}}$ when all bin auto- and cross-correlations are taken into account.

In an analysis only including auto-correlations the covariance matrix \eqref{eq:err-clgt} must be first reduced to the relevant components, and subsequently inverted. We denote the number counts angular power spectrum of a model neglecting lensing convergence by $\widetilde{C}_{\ell}$. The shift in the best-fit values due to neglecting lensing convergence is estimated through the systematic error \cite{Knox:1998fp,Heavens:2007ka,Kitching:2008eq,Camera:2014sba}
\begin{equation}
\Delta C_{\ell}=C^{\obs}_{\ell}-\widetilde{C}_{\ell}
\end{equation}  
where $C^{\obs}_{\ell}$ consistently includes lensing convergence. The bias induced by incomplete modelling of number counts angular power spectrum is then given by
\begin{equation}
\label{eq:shift}
\Delta_{\theta_{\alpha}}=\sum_{\beta} \left[\left(\widetilde{F}\right)^{-1}\right]_{\alpha\beta} B_{\beta} \;,
\end{equation}
where
\begin{equation}
B_{\beta} \equiv \sum_{(ij)(pq)} \sum_{\ell} \Delta C_{\ell}^{ ij} \frac{\partial \widetilde{C}_{\ell}^{pq}}{\partial \theta_{\beta}} {\rm Cov}_{\widetilde{C}_{\ell \, \rm[(ij), (pq)]}}^{-1} \;,
\end{equation}
and a tilde always denote the quantity computed according to the model neglecting lensing convergence.
This expression assumes that the systematic error does not affect the covariance. Moreover, it is only valid if the shifts \eqref{eq:shift} are small compared to the variances $\Delta^2_{\theta_{\alpha}}/\sigma^2_{\theta_{\alpha}}<1$.
The MCMC analysis shows that biases in cosmological parameters due to neglecting lensing convergence easily reach several standard deviations, so none of these assumptions is satisfied in our case.
Another point worth noticing is that Eq.~(\ref{eq:Fisher}) can only be used to estimate error contours by assuming that the underlying universe is described either by $C_{\ell}$ or by $\tilde{C}_{\ell}$, and does not give information about error contours obtained when  fitting the wrong model $\tilde{C}_{\ell}$ to data consistent with the full $C_{\ell}$ spectra. This signifies an advantage of the MCMC approach over the Fisher matrix technique, as shown in Chapter \ref{chapter-mnu}.

