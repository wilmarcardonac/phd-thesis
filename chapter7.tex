\chapter{Forecasts on neutrino masses with an Euclid-like survey}
\label{chapter:7}

\section{Introduction}

The study of the large scale structure of the universe allows us to test the different cosmological models. The sky coverage and the deepness in redshift of  galaxy surveys have improved significantly in the past decades. This, in addition to the high precision CMB experiments, will able us to break the degeneracies in cosmological models that have plagued cosmology for quiet a while already. In particular, a better mapping of the matter distribution in the universe will help to put tighter constraints on the masses of neutrinos. Indeed, the clustering properties of the mass in the universe depend on the neutrino masses and possibly on their hierarchy. 

Due to the experiments carried out in particle accelerators we know that neutrinos are massive. Nevertheless, we do not have a precise estimation for their masses which, in turn, unable us to estimate their total contribution to the energetic composition of the universe. Therefore, by measuring precisely the neutrino masses we can discriminate between different cosmological models since these predict distinct clustering properties. 

In this chapter we do forecasts on the neutrino masses by using an Euclid-like survey. We use formalism of the angular power spectrum of the matter distribution and include the relevant relativistic effects, namely, lensing and redshift space distortions. The advantage of this formalism over the usual power spectrum $P(k;\, z=z_0)$ is that \dots \texttt{I must have a review of the literature and understand much better the two formalisms. A first advantage is related to the fact that the $C_{\ell}$ formalism deals with quantities which are directly observable without any model assumption. \\
In the Euclid teleconference of Monday 14.12.2014 a Italian guy mentioned a third formalism. \\
In particular, it would be important to understand why we include only the relativistic effects above. \\
In addition, I must clarify the differences between spectroscopy and tomographic surveys. \\
In connection with this last point, I should make clear what the power of Euclid will be.}

In summary, in this chapter we aim to answer the following questions. First, will Euclid be able to measure the neutrino masses ?. Second, what error do we expect for such a survey ?. Third, will we be able to distinguish between different hierarchies with a Euclid-like survey ? 
