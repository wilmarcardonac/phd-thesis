\chapter{General solutions}
\label{appendix2-ade}

In Sec.\ \ref{section:3} we study some limiting cases in the $ 4- $dimensional system (\ref{eq:de-cont})-(\ref{eq:de-eul}) for which dark matter and dark energy perturbations decouple from each other.  In this appendix we give some solutions which are a bit cumbersome to be written in the main body of the paper. 

On sub-horizon scales and during matter dominance, dark energy density perturbations are governed by Eq. (\ref{eq:pheno7}) whose full solution is

\bea
\label{eq:appendix:A1}
\delta_{de} & = & \(\frac{\ceff k}{\H}\)  \, a^{\alpha_8}\,\lcb A_5\,  J_{-\nu_5}\( \frac{2 \ceff k}{\H} \) + A_6\, J_{\nu_5}\( \frac{2 \ceff k}{\H} \) \rcb  \nonumber \\  
&+& \aa \beta_6\, a^{\alpha_7} \( \frac{\ceff k}{\H} \)^{\alpha_6} \,  J_{\nu_5}\( \frac{2 \ceff k}{\H} \) \, \tilde{{}_1F_2} \( \nu_6\,;\, \alpha_6\,, \nu_7\, ; -\frac{k^2 \ceff^2}{\H^2} \)   \nonumber\\ 
& - & \aa \beta_7 a^{\alpha_7} \( \frac{\ceff k}{\H} \)^{\alpha_{9}} J_{-\nu_5}\(  \frac{2\ceff k}{\H} \) \tilde{{}_1F_2} \( \nu_8\,;\, \alpha_{9}\,, \nu_{9}\, ; -\frac{k^2 \ceff^2}{\H^2} \)  \,,  
 \eea

\noindent where 

\bea 
\nu_5 & = & \frac{\sqrt{432\ceff^4 + 48\ff^2 + 72\ceff^2(-1+4\ff-6w) + 3(1+6w)^2 -8\ff(3+4\gg^2+ 18w)}}{2\sqrt{3}}\,,\\
\nu_6 & = & -\frac{5}{4} -\ff + n + \frac{3w-\nu_5}{2}\,,
\eea

\bea
\nu_7 & = & \nu_6 + 1\,,\\
\nu_8 & = &  -\frac{5}{4} -\ff + n + \frac{3w+\nu_5}{2}\,,\\
\nu_9 & = & \nu_8 + 1\,,\\
\alpha_6 & = & 1 - \nu_5\,, \\
\alpha_7 & = & -2(1 +\ff) +3w +n\,,\\
\alpha_8 & = &  -\ff + \frac{3w}{2} - \frac{3}{4}\,, \\
\alpha_{9} & = & 1 + \nu_5\,,
\eea

\bea
\beta_6 & = & \frac{1}{24} \frac{\ceff^{-2-4\ff+6w} k^{-4\ff+6w} }{H_0^{2-4\ff+6w}}   \Omega_m^{-1+2\ff-3w} \pi \delta_0 \csc \( \pi \nu_5\) \Gamma \( \alpha_6\) \Gamma \( \nu_6\) \Gamma \( \alpha_{9}\)\,,\\
\beta_7 & = & \frac{1}{24} \frac{\ceff^{-2-4\ff+6w} k^{-4\ff+6w} }{H_0^{2-4\ff+6w}}   \Omega_m^{-1+2\ff-3w} \pi \delta_0 \csc \( \pi \nu_5\) \Gamma \( \alpha_6\) \Gamma \( \nu_9\) \Gamma \( \alpha_{9}\)\,,
\eea

\noindent $ A_5$, $ A_6  $ are constants of integration and $ \tilde{{}_1F_2} $ stands for the regularized generalized hypergeometric function. The last two terms in Eq. (\ref{eq:appendix:A1}) are due to the external anisotropic stress.

On the other hand, for sub-sound horizon scales, during dark energy domination and without external contribution to the dark energy anisotropic stress ($\aa=0$), we find (by using Eq. (\ref{eq:pheno19})) for dark energy velocity perturbations
\begin{align}
\label{eq:appendix:A2}
V_{de} = & \frac{1}{2} \( \frac{x_3}{2} \)^{\frac{1-\alpha_4}{1+3w}} \Bigg\{\,
  \bigg[\, 6(c_s^2-w) + 1-\alpha_4 \,\bigg] \times
 \nonumber \\ 
 &\times \[ B_4 \, \Gamma\(\frac{2+3w-\alpha_4}{1+3w}\)   J_{-\nu_1}(x_3)   + A_4\, \Gamma\(\frac{3w+\alpha_4}{1+3w}\)  J_{\nu_1}(x_3) \] \nonumber \\
&+ \frac{x_3 (1+3w)}{4} \[ B_4\,\Gamma\(\frac{2+3w-\alpha_4}{1+3w}\) \[ J_{-1-\nu_1}(x_3) - J_{1-\nu_1}(x_3)  \]    \right.
\nonumber \\
&+ \left. A_4\, \Gamma\(\frac{3w+\alpha_4}{1+3w}\) \[ J_{-1+\nu_1}(x_3) - J_{1+\nu_1}(x_3) \]    \]       \,\Bigg\} \, , 
\end{align}
%
while for dark matter velocity perturbations 
%
\bea 
\label{eq:appendix:A3}
V_m & = & \frac{3 (1+2\ff)}{1+3w} 2^{-\frac{2-\alpha_4 + \nu_1 +3w(1+\nu_1)}{1+3w}} x_3^{-\frac{-1+\alpha_4+\nu_1+3w\nu_1}{1+3w}} \,\Bigg\{
A_4 \, x_3^{2\nu_1}
 \\
& \times & \Gamma\(\frac{3w+\alpha_4}{1+3w}\) \Gamma\( \frac{2-\alpha_4 +3w(\nu_1 -1)+\nu_1}{2+6w} \)
\nonumber \\ 
&\times & {}_1F_2\( \lcb \frac{2-\alpha_4 + 3w(\nu_1-1) + \nu_1}{2+6w} \rcb,\, \lcb 1+\nu_1,\, \frac{4-\alpha_4+\nu_1+3w(1+\nu_1)}{2+6w}  \rcb,\, -\frac{1}{4}x_3^2  \)
 \nonumber \\
&+& B_4\, 4^{\nu_1}\, \Gamma\( \frac{2+3w-\alpha_4}{1+3w} \) \Gamma\(-\frac{-2+\alpha_4 + \nu_1 +3w(1+\nu_1)}{2+6w} \)
\nonumber \\
&\times & {}_1F_2\( \lcb \frac{2-\alpha_4 - 3w(\nu_1+1) - \nu_1}{2+6w} \rcb,\, \lcb 1-\nu_1,\, \frac{4-\alpha_4-\nu_1-3w(-1+\nu_1)}{2+6w}  \rcb,\, -\frac{1}{4}x_3^2  \)  \,\Bigg\}
\nonumber \, ,
\eea
%
where $ \alpha_4,\, \nu_1  $ and $ x_3 $ are given by Eq. (\ref{eq:pheno20}). By means of this solution we can easily find an expression for dark matter density perturbations by solving the differential equation 
\be
\delta_m' = \frac{V_m}{a} \,.
\label{eq:appendix:A4}
\ee